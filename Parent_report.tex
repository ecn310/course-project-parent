\documentclass[12pt]{article}

% set margins and spacing
\addtolength{\textwidth}{1.3in}
\addtolength{\oddsidemargin}{-.65in} %left margin
\addtolength{\evensidemargin}{-.65in}
\setlength{\textheight}{9in}
\setlength{\topmargin}{-.5in}
\setlength{\headheight}{0.0in}
\setlength{\footskip}{.375in}
\renewcommand{\baselinestretch}{1.0}
\linespread{1.0}

% load miscellaneous packages
\usepackage{csquotes}
\usepackage[american]{babel}
\usepackage[usenames,dvipsnames]{color}
\usepackage{graphicx,amsbsy,amssymb, amsmath, amsthm, MnSymbol,bbding,times, verbatim,bm,pifont,pdfsync,setspace,natbib}

% enable hyperlinks and table of contents
\usepackage[pdftex,
bookmarks=true,
bookmarksnumbered=false,
pdfview=fitH,
bookmarksopen=true,hyperfootnotes=false]{hyperref}

% define environments
\newtheorem{definition}{Definition}
\newtheorem{fact}{Fact}
\newtheorem{result}{Result}
\newtheorem{proposition}{Proposition}



\title{Intergenerational Mobility and Parental Time Investment: A State-Level Analysis}
\author{Aisling Gilmartian\thanks{Akgilmar@syr.edu} \and Devon Mitchell\thanks{Djmitche@syr.edu} \and Yvonne Achancho\thanks{Yeachanc@syr.edu} \and Oliver Levesque\thanks{Oalevesq@syr.edu}}
\date{\vskip-.1in \today}
\begin{document}
\maketitle

\vskip.3in
\begin{center} {\bf Abstract} \end{center}

\begin{quote}
{\small Insert abstract text here: 75-200 words, very high-level summary of your project. Look at the three most closely-related papers in your literature review to get an idea of what should be included.}
\end{quote}

\bigskip
\section{Introduction} \label{sec:introduction}

The concept of intergenerational mobility is a real and distinguishable factor of society that has been imprinted in many Americans since birth. The idea of "The American Dream" in essence is intergenerational mobility as many people come to this country in search of a better life for not just them, but their kids as well. They similarly do not want to make the life of their kids harder than theirs which reinforces the desire for intergenerational mobility to be upward. Parental investments are widely perceived to help positively affect this upward intergeneratioanl mobility. In this paper We investigate weather there is a relationship between the parental time investments of a state and the intergenerational mobility of that state. Our research question that summarizes what we are researching is do parents invest more time in their kids in U.S. states where there is more inequality and/or less intergenerational mobility? We will go about answering this question by utilizing data from The American Time Use Survey, which gives us observations from individual Americans who filled out a survey regarding
how they spent their days and gives us variables on their time invested in their children. Opportunity Insights, which gives us data regarding the intergenerational mobility of each state utilizing data of income for 24 year old children born in the 25th and 75th income percentiles. Finally, The US Census Bureau gives us our data on median income of each state which we used to test our theory on the two variables we are testing. We will use this data to quantify parental investment and intergenerational mobility variables which will help us determine the relationship between the two variables we have. 

From this data, we hypothesize that that Parents in U.S. states with higher rates of intergenerational mobility invest more time in their children compared to parents in states with lower rates of intergenerational mobility. Our research hypothesis is that Parents in U.S. states with higher rates of intergenerational mobility invest more time in their children compared to parents in states with lower rates of intergenerational mobility. The theory we have to support our hypothesis is that areas with higher intergenrational mobility, it is more likely for parents to focus on their parental time investments. This is because parents do not want their kids to drop in wealth and want to see upward movements in intergenerational mobility. Therefore, parents are incentivized to invest more time in their kids than states with less intergenerational mobility. We researched this by investigating median income of each state compared to time investments and intergenerational mobility of each state to determine if wealthier states played a role in parental time investments.  We found that there is a marginally statistically significant relationship between intergenerational mobility and parental investments by state. This indicates that there is some relationship between the two variables we study and would hint at parents wanting to While these results are not exactly what we hypothesized, It is still an important outcome of our research because it can inform parents that the parental investments that they make has some possible correlation on the intergenerational mobility of the state they live in. 

This paper begins with our literature review where we evaluate similar papers in the field to establish exactly where the research on our topic is missing which is parental time investments compared to intergenerational mobility as there is no exact paper in the field that exists in the manner we conducted our research. The next section of the paper is a description of our data and how we acquired, merged and used it to test our hypothesis. Our results section presents the empirical outcomes, using graphs and t-tests to summarize the data and provide evidence in support of our hypothesis that intergenerational mobility and parental time investments are positively corelated. Finally, our conclusion section discusses our entire process and the implications of our results in the real world. It also discusses the next steps in the field of research that should be taken. 

\section{Literature Review} \label{sec:literature}

These articles address how structural factors such as income inequality, parental socioeconomic status, and regional context shape parental time and investments in their children’s development. Schneider (2018) analyses and shares his findings that state that higher local income inequality increases class gaps in financial investments in children. This also affects time investments on both ends between parents and children because higher income sometimes leads to more freedom and parental participation in the children’s lives. This aligns with our findings as it demonstrates that parental time spent with children is influenced by certain factors such as child characteristics rather than being uniform across families. Nicoletti (2020) investigates how parents adjust time spent with their children based on the child’s current human capital such as compensating for poor cognitive skills by increasing learning time. Thus this aligns with our results which demonstrates that parents in high-inequality areas invest more time and resources in their children expecting high returns for their investments. This study aligns with Agostinelli, Doepke, Sorrenti and Zilibotti (2023), which shows that when children have academically stronger friends, nonauthoritarian parents increase their parental investment time with their children to make up for their peer skills. On the other hand, authoritarian parents don’t change their parental investment time but they do tend to control who their children surround themselves with. All of this ties into the Doepke's (2019) framework, which analyzes how parenting styles change according to economic conditions. The study explains that authoritative parents often invest more time in soft qualities such as patience and long-term focus when their children excel academically. Authoritarian parents are more dependent on restricting their children’s choices than increasing parental investment time. These sources show that inequality and environment don't just change how much parents invest, but also how they parent their children.

A common theme in these articles is the discussion of early investment in children and how it affects the trajectory of the child’s life. Carroll et al. (2014) adds to existing research about human development and mobility, and supports evidence of the importance of early life conditions in shaping life skills, some of which can be propelled by investment. Carroll tested this theory by using an economic equation that takes into account parental budget constraints. They modified the model to make it more robust by formulating his model through parental preferences and human development, which he admits narrows down the data. The result is an argument that supports the need for childhood investment at some capacity, which is paramount for the development of skills needed for a child's success, which can be translated into upward mobility. Carroll (2023) argues that a more contextualized approach to his earlier work. He argues that there are different sensitive periods in a child’s life and the importance of investing during these different periods to strengthen development. This is in contrast to Del Bono (2016), which concludes the importance of childhood investment but focuses on maternal time investment rather than resources, which goes through a different model that accounts for income and expected lifetime wealth (Carrol, 2023).

\section{Data}
\label{sec:data}


\subsection{Data Overview} \label{sec:literature}

We employ two datasets to test our hypothesis. First, the American Time Use Survey (ATUS) data set gives us the variables we need regarding parental time investments in each of the 50 states and Washington D.C. Second, the Opportunity Insights data set gives us variables about the intergenerational mobility of  the same 50 states and Washington D.C. These two data sets help us answer our question about time investments and intergenerational mobility.

\subsection{The American Time Use Survey}

The \href{https://www.bls.gov/tus/}{American Time Use Survey} collects data on how individuals spend their time in a given day. This data set was generated by randomly chosen households who had completed eight months of interviews with the Current Population Survey. This survey was done yearly from 2003 until 2025. We specifically choose the data from the years 2013-2014 as it overlaps with the Opportunity Insights data set that we are using. Then a time diary was given to them in which they recorded how they spent their day, each activity, and for how long. These activity's being the 792 variables covered in the raw data set but we only use 1 variable in our final analysis, that being total time spent providing secondary childcare. The scope is the United States, and there are 40,903 individual entries from the 50 States and Washington, DC. It also has the individual data that we need to aggregate up to the state level from the CPS portion of the ATUS. We merged the ATUS CPS data set and the Opportunity Insights data together on the state level using federal state FIPS codes that were in both datasets. We needed this merger to occur because it helps us link the parental investment variables from the ATUS data to the mobility variables from the Opportunity Insights data.

\subsection{The Opportunity Atlas}
\href{https://www.opportunityatlas.org/}{The Opportunity Atlas} is a collaboration between the U.S. Census Bureau and Opportunity Insights at Harvard University. The data set is generated by taking information from three sources. These sources are the 2000 and 2010 Decennial Census short form, the federal income tax returns from 1989, 1994, 1995, and 1998-2015, the 2000 Decennial Census long form, and the 2005-2015 American Community Surveys.

This data set contains average survey responses at the county level, encompassing  counties from every state in the United States. It includes 20 variables, such as county identification codes, children's national income rankings, college attendance rates among children, and state identification markers. The data set comprises 28,242 observations that span 50 states and D.C. 

The data covers 3,224 counties throughout the United States, but we aggregate this up to the state level here. We then take the mean of each county in each state weighted by the population of each county, which helps us to merge the Opportunity Insights data with the ATUS data.
We chose 2014 as the targeted year to cross section the Opportunity Insights data with the ATUS data because the Opportunity Insights data set had very limited data to work with as it only covered 2014. The Opportunity Insights data we chose that considered variables such as reported income and birth cohort. We found these variables necessary for our research despite not having multiple years of data to work with.
    
\subsection{Merged data}
\label{sec:data}

We have Compiled all the Equality of Opportunity income data from 2014, Organized ATUS data and appended all of the microfiles together along with the CPS data which has participants identify their state Fips code which shows the federal fips code of each state. We then used the Fips code to merge the ATUS data along with the Equality of opportunity data because it also contains the Fips code for the state that each county is in. What we ended up with was a fully combined data set with the variable of child income at age 24 for children born in the 25th and 75th percentiles as our indicators of intergenerational mobility in each state. 
We merged our data by using the statefips variable and after taking the mean of every county in the Equality of Opportunity data set, we merged the combined ATUS data set with the Opportunity Insights data set using the statefips variable.



\begin{tabular}{lccccc} \hline
 & (1) & (2) & (3) & (4) & (5) \\
VARIABLES & N & mean & sd & min & max \\ \hline
 &  &  &  &  &  \\
Total time spent providing secondary childcare  & 51 & 182.38 & 37.42 & 90.61 & 318.15 \\
Rank in national income at age 24 for child born in 25th  & 51 & 32.39 & 12.15 & 6.98 & 49.827 \\
Rank in national income at age 24 for child born in 75th & 51 & 38.40 & 14.44 & 7.58 & 58.36 \\
Median income of each state & 51 & 54131.2 & 9005.30 & 39464 & 74149 \\
 &  &  &  &  &  \\ \hline
\end{tabular}



\section{Results}
The analysis we conducted was between our established variables for parental time investments which records the amount of time a parent spent in their daily diary entry on childcare, and our variables for two intergenerational mobility, which measures the amount of income that a resident born into the 25th and 75th percentile of income distribution has at age 24, aggregated to the state level. 
Our research hypothesis is that parents in U.S. states with higher rates of intergenerational mobility invest more time in their children compared to parents in states with lower rates of intergenerational mobility. Our null hypothesis is that there is no relationship between intergenerational mobility of states and parental time investments. 

We then had to create a new variable of high mobility, which divided our data into the 26 highest-mobility states to prepare the data for a t-test
We performed a two sample t-test and found that the difference in means between the the higher parental investment states and the the lower parental investment states was 18.48 minutes per day, with the higher-mobility group having the higher average childcare per day.

For the 25 percentile measure, our p value was 0.1413. Because this is outside both the .05 and .1 significance thresholds, there is little evidence of a relationship. Additionally, our confidence interval of ($-3.25$, $38.10$) contains zero, so we cannot be certain this difference is not due to chance. Thus, based this t-test, we find that the data we have do not show a statistically significant relationship between parental investment and intergenerational mobility by state, and we fail to reject the null hypothesis.


\begin{center}
  \includegraphics[height=6cm]{Reproducibility Package/Outputs/25novar.png}
\end{center}

\begin{center}
  \includegraphics[height=6cm]{Reproducibility Package/Outputs/75novar.png}
\end{center}

The graph for these p-values is the plot of parental investment versus intergenerational mobility, with each point labeled by its corresponding \texttt{statefips} code.

Additionally, we performed another t-test to examine the relationship between parental time investments and intergenerational mobility for individuals born in the 75th percentile of the income distribution. This test provides insight into how impactful parental time investments may be for wealthier individuals. We followed the same procedure as with the 25th-percentile variable: we split the 51 state-level observations at the median, created a new high-mobility variable, \texttt{hmob}, and conducted a two-sided t-test based on this new variable.

Our results showed a p-value of 0.0968 which is outside our accepted significance level of .05 meaning our results are not statistically significant and we fail to reject our null hypothesis that parental time investments has no impact on intergenerational mobility for children born in the 75th percentile.
\
-7U8y0
\begin{center}
  \includegraphics[height=6cm]{Reproducibility Package/Outputs/Parental Investment vs median income of state.jpg}
\end{center}

We also created a scatter plot of the average income by state for the year 2014, including all participants in the U.S. Census Bureau data rather than only children's data at the age of 24. This allows us to show that the income data for 24-year-olds is not fully representative of the overall population.


    \setlength{\pdfpagewidth}{8.5in} \setlength{\pdfpageheight}{11in}


   

\section{Discussion}
\label{sec:discussion}

Optional. This is where you would discuss any of the following if they are extensive; otherwise, this material can be put in the Results (caveats) and Conclusions (future work and next steps) sections.
\begin{itemize}
    \item caveats (are there problems with the data that there are no obvious ways to resolve? if so, how might this impact your results?)
    \item future work / next steps
    \item implications of the results: that is, how your findings -- if they were causally identified -- might inform policymaking, etc.
\end{itemize}

\section{Conclusion}
\label{sec:conclusion}

We began our research with the question of Do parents invest more time in their kids in U.S. states where there is more inequality and/or less intergenerational mobility? This led us to researching the topic with data from The American Time Use Survey for data regarding how parents invest time in their children, and data from Opportunity Insights to gather data on intergrenerational mobility. We aggregated this data up to the state level because the ATUS data could not be aggregated further than that point. We also weighted the data by the population of each state to better reflect the trends of the overall data set. With this data, and data we later gathered from the Census Bureau on median income of each state, we analyzed how high mobility states compare with low mobility states in their parental time investments. Our results were that there is a marginally statistically significant relationship between intergenerational mobility of a state and the parental time investments of that  state. This is because the tests we performed on the variables we researched had p values below .1. With these results, We can conclude that Mobility of a state correlates to the investments parents make in their children because of our marginally statistically significant conclusion. From this conclusion, the next steps would be to expand this research down to the county level to get more accurate data on intergenerational mobility. Additionally, individual data stating personal reasons why parents make the investments that they do in their children would be greatly beneficial as it would give a new perspective on this topic. The implications of our results are that parental investments and Intergenerational mobility of a state have a marginally statistically significant relationship which means that public policy changes that would create more intergenerational mobility in a state would give parents more incentives to invest in their children.
\newpage
\section*{Bibliography}
\singlespacing
\setlength\bibsep{0pt}
Agostinelli, F., Doepke, Sorrenti, G., \& Zilibotti, F. (2023). It Takes a Village: The Economics of Parenting with Neighborhood and Peer Effects. IZA - Institute of Labor Economics.
    
Carroll, Christopher D., et al. “The Distribution of Wealth and the Marginal Propensity to Consume.” National Bureau of Economic Research, Working Paper 19925, Feb. 2014

Carroll, Christopher D., et al. "Wealth Inequality and the Marginal Propensity to Consume." National Bureau of Economic Research, Working Paper 31093, Mar. 2023,

Del Bono, Emilia, et al. “Early Maternal Time Investment and Early Child Outcomes.” The Economic Journal, vol. 126, no. 596, Oct. 2016, pp. F96–F135. JSTOR, doi:10.1111/ecoj.12342

Doepke, Matthias, et al. "The Economics of Parenting." Annual Review of Economics, vol. 11, 2019, pp. 55-84. Annual Reviews, https://doi.org/10.1146/annurev-economics-080218-030156.

Nicoletti, C., \& Tonei, V. (2020). Do parental time investments react to changes in child’s skills and health? European Economic Review, 127, 103491. https://doi.org/10.1016/j.euroecorev.2020.103491

Schneider, Daniel, et al. "Income Inequality and Class Divides in Parental Investments." American Sociological Review, vol. 83, no. 3, 2018, pp. 475-507.

\newpage
\section*{Data Appendix} \label{sec:appendixa}
\addcontentsline{toc}{section}{Appendix A}

You should at least direct your reader to your replication package. You might put key elements of your replication package in this section as well.

\end{document}
