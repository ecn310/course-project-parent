\documentclass[12pt]{article}

% set margins and spacing
\addtolength{\textwidth}{1.3in}
\addtolength{\oddsidemargin}{-.65in} %left margin
\addtolength{\evensidemargin}{-.65in}
\setlength{\textheight}{9in}
\setlength{\topmargin}{-.5in}
\setlength{\headheight}{0.0in}
\setlength{\footskip}{.375in}
\renewcommand{\baselinestretch}{1.0}
\linespread{1.0}

% load miscellaneous packages
\usepackage{csquotes}
\usepackage[american]{babel}
\usepackage[usenames,dvipsnames]{color}
\usepackage{graphicx,amsbsy,amssymb, amsmath, amsthm, MnSymbol,bbding,times, verbatim,bm,pifont,pdfsync,setspace,natbib}

% enable hyperlinks and table of contents
\usepackage[pdftex,
bookmarks=true,
bookmarksnumbered=false,
pdfview=fitH,
bookmarksopen=true,hyperfootnotes=false]{hyperref}

% define environments
\newtheorem{definition}{Definition}
\newtheorem{fact}{Fact}
\newtheorem{result}{Result}
\newtheorem{proposition}{Proposition}



\title{ECN 310: Parent}
\author{Aisling Gilmartian\thanks{Akgilmar@syr.edu} \and Devon Mitchell\thanks{Djmitche@syr.edu} \and Yvonne Achancho\thanks{Yeachanc@syr.edu} \and Oliver Levesque\thanks{Oalevesq@syr.edu}}
\date{\vskip-.1in \today}
\begin{document}
\maketitle

\vskip.3in
\begin{center} {\bf Abstract} \end{center}

\begin{quote}
{\small Insert abstract text here: 75-200 words, very high-level summary of your project. Look at the three most closely-related papers in your literature review to get an idea of what should be included.}
\end{quote}

\bigskip
\section{Introduction} \label{sec:introduction}

Answer the questions
\begin{enumerate}
    \item \textbf{Why should the reader care? / Why is the topic important?} (required)
    \item \textbf{What question will you answer? How will you do it?} (required)
        \begin{enumerate}
            \item If your theory/hypothesis fit in one paragraph, include it here. If it is longer, make it a separate section after the lit review. EITHER OPTION IS FINE as long as the length is sufficient/appropriate for your project.
        \end{enumerate}
    \item \textbf{What did you find?} (required)
    \item \textbf{Give a "road map" of the paper. Where will the reader find the various parts of your work?} (required)
\end{enumerate}

\section{Literature Review} \label{sec:literature}

Discuss at least seven papers that are closely related to your results (more is better). Explain how they're related. Did you find something similar, or different? Did you look at a different context? Different time period? Different level of detail?
\begin{itemize}
    \item The most important thing to do here is to highlight how what you do compares to these other papers
\end{itemize}

\section{Theoretical Analysis}
\label{sec:theory}
Optional--may include in intro if it's short.


\section{Data}
\label{sec:data}


\subsection{Data Overview} \label{sec:literature}

We employ two datasets to test our hypothesis. First, the American Time Use Survey (ATUS) data set gives us the variables we need regarding parental time investments and weekly income of parents in each of the 50 states and Puerto Rico we are observing. Second, The Equality of Opportunity data set gives us variables about the intergenerational mobility of states and territories being that we are researching. These include income of children at age 24 born in the 25th and 75th percentiles of income. these two data sets help us answer our question regarding time investments and intergenerational mobility.

We employ two datasets to test our hypothesis. First, the Equality of Opportunity data set provides variables on the intergenerational mobility of each state and territory in the United States. Second, the American Time Use Survey (ATUS)  data set provides the variables we need on parental time investments for each of the 51 states and territories we studied. These two data sets help us answer our question regarding time investments and intergenerational mobility.

\subsection{The Equality of Opportunity}
 The Opportunity Insights is a  not-for-profit organization between Harvard, Brown, and the U.S. Census Bureau.\href{https://opportunityinsights.org/}{ Opportunity Insights}The data set is generated by taking information from three sources. Namely, The 2000 and 2010 Decennial Census short form, the federal income tax returns from 1989, 1994, 1995, and 1998-2015, as well as the 2000 Decennial Census long form, and 2005-2015 American Community Surveys.(depends on the specific variable, this ones comes from federal income tax records from the year 2014-2015). 
 
 
 
 Our observations the average of respondents in each US county, and the scope covers all counties within the U.S. There are 20 variables with some examples being County identification, rank in national income of children, college attendance by children, and state identification. There are a total of 28,242 Observations from 51 States and territories 

The data covers 3224 counties throughout the United States, but we aggregate this  up to the state level here. We then take the mean of each county in each state, which helps us to merge the Equality of Opportunity data with the ATUS data.
 We chose 2014 as the targeted year to cross section the Equality of Opportunity data with the ATUS data because the Equality of Opportunity data set we chose that considered variables such as county and birth cohort only provided data from 2014. We found these variables necessary for our research despite not having multiple years of data. 
    
\subsection{The American Time Use Survey}
The \href{https://www.bls.gov/tus/}{American Time Use Survey} contains the variables we need regarding individuals and how they spend their time in a given day with childcare. It also has the individual data that we need to aggregate up to the state level from the CPS portion of the ATUS. We utilized this common variable between the appended ATUS CPS data set and the Equality of opportunity data set to merge our data sets together on the state level using federal state FIPS codes. is taken by the U.S. Bureau of Labor Statistics. This data set was generated by randomly chosen households who had completed eight month of interviews with the Current Population Survey, then given a time diary on how they spent their past day within a time-use survey. The data covers all households within the U.S. of at least 15 years of age, the years we used cover 2013-2014 as it most accurately covered individuals willing to participate in the survey, but the ATUS data covers all years since 2003. The data from 2013-2014 is also the most available year to overlap this dataset with the Equality for Opportunity dataset as they collected data from respondents during the same time-frame. There are 792 variables covered in our do-file dataset. The scope is the The United States and there are 40,903 individual entries from 51 states and territories


\subsection{Merged data}
\label{sec:data}


We have Compiled all the Equality of Opportunity income data from 2014 into Stata. Organized ATUS data and appended all of the stata microfiles together along with the CPS data which has participants identify their state Fips code, we then used the Fips code to merge the ATUS data along with the Equality of opportunity data because it also contains the Fips code for the state that each county is in.  What we ended up with was a fully combined data set with the variable of child income at age 24 for children born in the 25th percentile as our indicator of intergenerational mobility in each state. 
We merged our data by Using the statefips variable which shows the federal fips code of each state and after taking the mean of every county in the Equality of Opportunity data set, we used the merge command on Stata to merge the combined ATUS data set with the Equality of Opportunity data set using the statefips variable.


\begin{tabular}{lccccc} \hline
 & (1) & (2) & (3) & (4) & (5) \\
VARIABLES & N & mean & sd & min & max \\ \hline
 &  &  &  &  &  \\
Total Time spent providing secondary childcare & 51 & 190.5 & 77.16 & 38.82 & 601.9 \\
Rank in national income at age 24 for child born in 25th & 51 & 48.14 & 4.241 & 39.00 & 56.36 \\
Rank in national income at age 24 for child born in 75th & 51 & 57.18 & 3.138 & 48.33 & 66.78 \\
 &  &  &  &  &  \\ \hline
\end{tabular}


Key Variables
Some key variables from the ATUS data set are: Weekly income per individual (trernwa) and total time in day spent on secondary childcare with household children (trthh). Some key Equality of Opportunity vairables are: Amount of income (in thousands) that a resident born in the 25th percentile of income has at age 24 (permresp25kr24) and State federal fips code (statefips).


\section{Results}
\begin{itemize}
    \item The analysis we did was between our established variables for parental time investments, \texttt{trthh}, which recorded the amount of time a parent spent in their daily diary entry for childcare and our variable for intergenerational mobility, \texttt{permresp25kr24}, which measures the amount of income (in thousands) that a resident born in the 25th percentile of income has at age 24 aggregated to the state level.
    \item We also used a \texttt{statefips} variable to measure which state each observation belonged to.
    \item Our research hypothesis is Parents in U.S. states with higher rates of intergenerational mobility invest more time in their children compared to parents in states with lower rates of intergenerational mobility and the null hypothesis we have is there is no relationship between intergenerational mobility of states and parental time investments.
    \item In order to merge the two data sets we had, we needed to take the mean of our \texttt{permresp25kr24} variable by state and we also took the mean of our \texttt{trthh} variable by state to compare the two variables.
    \item We then had to create a new variable of high mobility, \texttt{hmob}, which divided our data into the 26 highest states for intergenerational mobility to prepare our data for a t-test.
    \item We then performed a two sample t-test on our data and found that the difference in means of the higher parental investment states vs the lower parental investment states was 18.48 minutes per day, with the higher mobility group having the higher average childcare per day.
    \item Our p value was 0.078 and with it being between .05 and .1, there is some evidence of a relationship and to fail to reject our null hypothesis. However, our confidence interval of ($-39.08$, $2.13$) contains zero so we cannot be certain this difference is not purely due to chance.
    \item A graph of the parental investment vs intergenerational mobility that shows each \texttt{statefips} code is also generated through the .do file found on our GitHub repository.
    \item This same .do file also contains the code to recreate the t-test performed on the data.
    \item Through this t-test, we find that the data we have does not show a statistically significant relationship between parental investment and intergenerational mobility by state, and we fail to reject the null hypothesis.
    \item Additionally, we performed another t-test that examined the relationship between parental time investments in people born in the 75th percentile at age 24 which shows how impactful time investments are in wealthier individuals.
    \item We performed the same procedure as with the 25th percentile variable where we split the 51 observations by the median and created a new variable, highmob, this way and performed a two sided ttest based on this new variable
    \item Our results were that we had a p value of 0.6270 which is outside our accepted significance level of .05 meaning our results are not statistically significant and we fail to reject our null hypothesis that parental time investments has no impact on intergenerational mobility for children born in the 75th percentile.

    \setlength{\pdfpagewidth}{8.5in} \setlength{\pdfpageheight}{11in}

\begin{tabular}{lccccc} \hline
 & (1) & (2) & (3) & (4) & (5) \\
VARIABLES & N & mean & sd & min & max \\ \hline
 &  &  &  &  &  \\
Total Time spent providing secondary childcare & 51 & 190.5 & 77.16 & 38.82 & 601.9 \\
Rank in national income at age 24 for child born in 25th & 51 & 48.14 & 4.241 & 39.00 & 56.36 \\
Rank in national income at age 24 for child born in 75th & 51 & 57.18 & 3.138 & 48.33 & 66.78 \\
 &  &  &  &  &  \\ \hline
\end{tabular}
\end{itemize}
   

\section{discussion}
\label{sec:discussion}

Optional. This is where you would discuss any of the following if they are extensive; otherwise, this material can be put in the Results (caveats) and Conclusions (future work and next steps) sections.
\begin{itemize}
    \item caveats (are there problems with the data that there are no obvious ways to resolve? if so, how might this impact your results?)
    \item future work / next steps
    \item implications of the results: that is, how your findings -- if they were causally identified -- might inform policymaking, etc.
\end{itemize}

\section{Conclusion}
\label{sec:conclusion}

Re-state (in different words) what you did and what you learned. If your discussion (Section 6) would be short, you can just have a Conclusion section that includes your discussion (that is, leave out a separate Discussion section).

\newpage
\section*{Bibliography}
\singlespacing
\setlength\bibsep{0pt}

You can either explicitly include your list of references, or you can learn to use BibTex so that it includes the references automatically.

Either way, this list should include ONLY the papers (reports, book chatpers, etc.) that you actually cite in the text (no extra).

At the same time EVERYTHING you cite in the main text must have an entry here (no references in text that don't have something here).

You can choose which citation style to follow. Whichever you choose, you must follow it consistently, including the convention to always alphabetize by the first author's last name.





\setlength{\pdfpagewidth}{8.5in} \setlength{\pdfpageheight}{11in}

\begin{tabular}{lccccc} \hline
 & (1) & (2) & (3) & (4) & (5) \\
VARIABLES & N & mean & sd & min & max \\ \hline
 &  &  &  &  &  \\
Total Time spent providing secondary childcare & 51 & 190.5 & 77.16 & 38.82 & 601.9 \\
Rank in national income at age 24 for child born in 25th & 51 & 48.14 & 4.241 & 39.00 & 56.36 \\
Rank in national income at age 24 for child born in 75th & 51 & 57.18 & 3.138 & 48.33 & 66.78 \\
 &  &  &  &  &  \\ \hline
\end{tabular}

\newpage
\section*{Data Appendix} \label{sec:appendixa}
\addcontentsline{toc}{section}{Appendix A}

You should at least direct your reader to your replication package. You might put key elements of your replication package in this section as well.

\end{document}
