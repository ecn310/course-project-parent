\documentclass[12pt]{article}

% set margins and spacing
\addtolength{\textwidth}{1.3in}
\addtolength{\oddsidemargin}{-.65in} %left margin
\addtolength{\evensidemargin}{-.65in}
\setlength{\textheight}{9in}
\setlength{\topmargin}{-.5in}
\setlength{\headheight}{0.0in}
\setlength{\footskip}{.375in}
\renewcommand{\baselinestretch}{1.0}
\linespread{1.0}

% load miscellaneous packages
\usepackage{csquotes}
\usepackage[american]{babel}
\usepackage[usenames,dvipsnames]{color}
\usepackage{graphicx,amsbsy,amssymb, amsmath, amsthm, MnSymbol,bbding,times, verbatim,bm,pifont,pdfsync,setspace,natbib}

% enable hyperlinks and table of contents
\usepackage[pdftex,
bookmarks=true,
bookmarksnumbered=false,
pdfview=fitH,
bookmarksopen=true,hyperfootnotes=false]{hyperref}

% define environments
\newtheorem{definition}{Definition}
\newtheorem{fact}{Fact}
\newtheorem{result}{Result}
\newtheorem{proposition}{Proposition}



\title{ECN 310: Parent}
\author{Aisling Gilmartian\thanks{Akgilmar@syr.edu} \and Devon Mitchell\thanks{Djmitche@syr.edu} \and Yvonne Achancho\thanks{Yeachanc@syr.edu} \and Oliver Levesque\thanks{Oalevesq@syr.edu}}
\date{\vskip-.1in \today}
\begin{document}
\maketitle

\vskip.3in
\begin{center} {\bf Abstract} \end{center}

\begin{quote}
{\small Insert abstract text here: 75-200 words, very high-level summary of your project. Look at the three most closely-related papers in your literature review to get an idea of what should be included.}
\end{quote}

\bigskip
\section{Introduction} \label{sec:introduction}

Answer the questions
\begin{enumerate}
    \item \textbf{Why should the reader care? / Why is the topic important?} (required)
    \item \textbf{What question will you answer? How will you do it?} (required)
        \begin{enumerate}
            \item Our research hypothesis is that Parents in U.S. states with higher rates of intergenerational mobility invest more time in their children compared to parents in states with lower rates of intergenerational mobility.
            \item The theory we have to support this is that states with higher intergenerational mobility tend to have more impact on parental time investments because parents do not want their kids to drop in wealth and are therefore incentivized to invest more time in their kids than states with less intergenerational mobility. 
            \item We researched this by investigating median income of each state compared to time investments and intergenerational mobility of each state to determine if wealthier states played a role in parental time investments. 
        \end{enumerate}
    \item \textbf{What did you find?} (required)
    \item \textbf{Give a "road map" of the paper. Where will the reader find the various parts of your work?} (required)
\end{enumerate}

\section{Literature Review} \label{sec:literature}

Discuss at least seven papers that are closely related to your results (more is better). Explain how they're related. Did you find something similar, or different? Did you look at a different context? Different time period? Different level of detail?
\begin{itemize}
    \item The most important thing to do here is to highlight how what you do compares to these other papers
\end{itemize}

\section{Theoretical Analysis}
\label{sec:theory}
Optional--may include in intro if it's short.


\section{Data}
\label{sec:data}


\subsection{Data Overview} \label{sec:literature}

We employ two datasets to test our hypothesis. First, the American Time Use Survey (ATUS) data set gives us the variables we need regarding parental time investments in each of the 50 states and Washington D.C. Second, the Opportunity Insights data set gives us variables about the intergenerational mobility of  the same 50 states and Washington D.C. These two data sets help us answer our question about time investments and intergenerational mobility.

\subsection{The American Time Use Survey}

The\href{https://www.bls.gov/tus/}{American Time Use Survey} contains the variables we need regarding individuals and how they spend their time in a given day doing childcare. This data set was generated by randomly chosen households who had completed eight months of interviews with the Current Population Survey. This survey was done yearly from 2003 until 2025. We specifically choose the data from the years 2013-2014 as it overlaps with the Opportunity Insights data set that we are using. Then a time diary was given to them in which they recorded how they spent their day, each activity, and for how long. These activity's being the 792 variables covered in the raw data set but we only use 1 variable in our final analysis, that being total time spent providing secondary childcare. The scope is the United States, and there are 40,903 individual entries from the 50 States and DC. It also has the individual data that we need to aggregate up to the state level from the CPS portion of the ATUS. We merged the ATUS CPS data set and the Opportunity Insights data together on the state level using federal state FIPS codes that were in both datasets. We needed this merger to occur because it helps us link the parental investment variables from the ATUS data to the mobility variables from the Opportunity Insights data.

\subsection{The Opportunity Atlas}
\href{https://www.opportunityatlas.org/}{The Opportunity Atlas} is collaboration between the U.S. Census Bureau and Opportunity Insights at Harvard University. . The data set is generated by taking information from three sources. These sources are the 2000 and 2010 Decennial Census short form, the federal income tax returns from 1989, 1994, 1995, and 1998-2015, the 2000 Decennial Census long form, and the 2005-2015 American Community Surveys.

This data set contains average survey responses at the county level, encompassing  counties from every state in the United States. It includes 20 variables, such as county identification codes, children's national income rankings, college attendance rates among children, and state identification markers. The data set comprises 28,242 observations that span 50 states and D.C. 

The data covers 3224 counties throughout the United States, but we aggregate this up to the state level here. We then take the mean of each county in each state weighed by the population of each county, which helps us to merge the Opportunity Insights data with the ATUS data.
We chose 2014 as the targeted year to cross section the Opportunity Insights data with the ATUS data because the Opportunity Insights data set had very limited data to work with as it only covered 2014. The Opportunity Insights data we chose that considered variables such as reported income and birth cohort. We found these variables necessary for our research despite not having multiple years of data to work with.
    
\subsection{Merged data}
\label{sec:data}

We have Compiled all the Equality of Opportunity income data from 2014, Organized ATUS data and appended all of the microfiles together along with the CPS data which has participants identify their state Fips code which shows the federal fips code of each state. We then used the Fips code to merge the ATUS data along with the Equality of opportunity data because it also contains the Fips code for the state that each county is in. What we ended up with was a fully combined data set with the variable of child income at age 24 for children born in the 25th and 75th percentiles as our indicators of intergenerational mobility in each state. 
We merged our data by using the statefips variable and after taking the mean of every county in the Equality of Opportunity data set, we merged the combined ATUS data set with the Opportunity Insights data set using the statefips variable.



\begin{tabular}{lccccc} \hline
 & (1) & (2) & (3) & (4) & (5) \\
VARIABLES & N & mean & sd & min & max \\ \hline
 &  &  &  &  &  \\
Total time spent providing secondary childcare  & 51 & 182.38 & 37.42 & 90.61 & 318.15 \\
Rank in national income at age 24 for child born in 25th  & 51 & 32.39 & 12.15 & 6.98 & 49.827 \\
Rank in national income at age 24 for child born in 75th & 51 & 38.40 & 14.44 & 7.58 & 58.36 \\
Median income of each state & 51 & 54131.2 & 9005.30 & 39464 & 74149 \\
 &  &  &  &  &  \\ \hline
\end{tabular}



\section{Results}
The analysis we did was between our established variables for parental time investments, \texttt{trthh}, which recorded the amount of time a parent spent in their daily diary entry for childcare and our variables for intergenerational mobility, \texttt{permresp25kr24} and \texttt{permresp75kr24}, which measures the amount of income (in thousands) that a resident born into the 25th and 75th percentile of income has at age 24 aggregated to the state level. Our research hypothesis is parents in U.S. states with higher rates of intergenerational mobility invest more time in their children compared to parents in states with lower rates of intergenerational mobility. Our null hypothesis we have is there is no relationship between intergenerational mobility of states and parental time investments. In order to merge the two data sets we aggregated the intergenerational mobility variables,  \texttt{permresp25kr24} and \texttt{permresp25kr24}, from county averages to state average and weighed them by their county popultaions. For the time investment variable,  \texttt{trthh}, we aggregated it from individual time diaries to state level.
We then had to create a new variable of high mobility, \texttt{hmob}, which divided our data into the 26 highest states for intergenerational mobility to prepare our data for a t-test.
We then performed a two sample t-test on our data and found that the difference in means of the higher parental investment states vs the lower parental investment states was 18.48 minutes per day, with the higher mobility group having the higher average childcare per day.

For the 25 precentile our p value was 0.1413 and with it being outside  .05 and .1, there is little evidence of a relationship. Additional our confidence interval of ($-3.25$, $38.10$) contains zero so we cannot be certain this difference is not purely due to chance. So through this t-test, we find that the data we have does not show a statistically significant relationship between parental investment and intergenerational mobility by state, and we fail to reject the null hypothesis.


\begin{center}
  \includegraphics[height=6cm]{}
\end{center}


The graph for these P vales it the parental investment vs intergenerational mobility that shows each \texttt{statefips} code. 

Additionally, we performed another t-test that examined the relationship between parental time investments in people born in the 75th percentile at age 24 which shows how impactful time investments are in wealthier individuals.
We performed the same procedure as with the 25th percentile variable where we split the 51 observations by the median and created a new variable, highmob, this way and performed a two sided ttest based on this new variable
Our results were that we had a p value of 0.6270 which is outside our accepted significance level of .05 meaning our results are not statistically significant and we fail to reject our null hypothesis that parental time investments has no impact on intergenerational mobility for children born in the 75th percentile.

    \setlength{\pdfpagewidth}{8.5in} \setlength{\pdfpageheight}{11in}


   

\section{discussion}
\label{sec:discussion}

Optional. This is where you would discuss any of the following if they are extensive; otherwise, this material can be put in the Results (caveats) and Conclusions (future work and next steps) sections.
\begin{itemize}
    \item caveats (are there problems with the data that there are no obvious ways to resolve? if so, how might this impact your results?)
    \item future work / next steps
    \item implications of the results: that is, how your findings -- if they were causally identified -- might inform policymaking, etc.
\end{itemize}

\section{Conclusion}
\label{sec:conclusion}

Re-state (in different words) what you did and what you learned. If your discussion (Section 6) would be short, you can just have a Conclusion section that includes your discussion (that is, leave out a separate Discussion section).

\newpage
\section*{Bibliography}
\singlespacing
\setlength\bibsep{0pt}

You can either explicitly include your list of references, or you can learn to use BibTex so that it includes the references automatically.

Either way, this list should include ONLY the papers (reports, book chatpers, etc.) that you actually cite in the text (no extra).

At the same time EVERYTHING you cite in the main text must have an entry here (no references in text that don't have something here).

You can choose which citation style to follow. Whichever you choose, you must follow it consistently, including the convention to always alphabetize by the first author's last name.


\begin{center}
  \includegraphics[height=5cm]{Reproducibility Package/Outputs/Parental Investment vs Intergenerational Mobility 75.png}
\end{center}


\begin{center}
  \includegraphics[height=5cm]{Reproducibility Package/Outputs/Parental Investment vs median income of state.jpg}
\end{center}





\newpage
\section*{Data Appendix} \label{sec:appendixa}
\addcontentsline{toc}{section}{Appendix A}

You should at least direct your reader to your replication package. You might put key elements of your replication package in this section as well.

\end{document}
