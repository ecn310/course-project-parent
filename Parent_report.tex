\documentclass[12pt]{article}

% set margins and spacing
\addtolength{\textwidth}{1.3in}
\addtolength{\oddsidemargin}{-.65in} %left margin
\addtolength{\evensidemargin}{-.65in}
\setlength{\textheight}{9in}
\setlength{\topmargin}{-.5in}
\setlength{\headheight}{0.0in}
\setlength{\footskip}{.375in}
\renewcommand{\baselinestretch}{1.0}
\linespread{1.0}

% load miscellaneous packages
\usepackage{csquotes}
\usepackage[american]{babel}
\usepackage[usenames,dvipsnames]{color}
\usepackage{graphicx,amsbsy,amssymb, amsmath, amsthm, MnSymbol,bbding,times, verbatim,bm,pifont,pdfsync,setspace,natbib}

% enable hyperlinks and table of contents
\usepackage[pdftex,
bookmarks=true,
bookmarksnumbered=false,
pdfview=fitH,
bookmarksopen=true,hyperfootnotes=false]{hyperref}

% define environments
\newtheorem{definition}{Definition}
\newtheorem{fact}{Fact}
\newtheorem{result}{Result}
\newtheorem{proposition}{Proposition}



\title{ECN 310: Parent}
\author{Aisling Gilmartian\thanks{Akgilmar@syr.edu} \and Devon Mitchell\thanks{Djmitche@syr.edu} \and Yvonne Achancho\thanks{Yeachanc@syr.edu} \and Oliver Levesque\thanks{Oalevesq@syr.edu}}
\date{\vskip-.1in \today}
\begin{document}
\maketitle

\vskip.3in
\begin{center} {\bf Abstract} \end{center}

\begin{quote}
{\small Insert abstract text here: 75-200 words, very high-level summary of your project. Look at the three most closely-related papers in your literature review to get an idea of what should be included.}
\end{quote}

\bigskip
\section{Introduction} \label{sec:introduction}

Answer the questions
\begin{enumerate}
    \item \textbf{Why should the reader care? / Why is the topic important?} (required)
    \item \textbf{What question will you answer? How will you do it?} (required)
        \begin{enumerate}
            \item If your theory/hypothesis fit in one paragraph, include it here. If it is longer, make it a separate section after the lit review. EITHER OPTION IS FINE as long as the length is sufficient/appropriate for your project.
        \end{enumerate}
    \item \textbf{What did you find?} (required)
    \item \textbf{Give a "road map" of the paper. Where will the reader find the various parts of your work?} (required)
\end{enumerate}

\section{Literature Review} \label{sec:literature}

Discuss at least seven papers that are closely related to your results (more is better). Explain how they're related. Did you find something similar, or different? Did you look at a different context? Different time period? Different level of detail?
\begin{itemize}
    \item The most important thing to do here is to highlight how what you do compares to these other papers
\end{itemize}

\section{Theoretical Analysis}
\label{sec:theory}
Optional--may include in intro if it's short.


\section{Data}
\label{sec:data}

\section{Research Question} \label{sec:question}

Do parents invest more time in their kids in U.S. states where there is more inequality and/or less inter-generational mobility?

\section{Data Overview} \label{sec:literature}

The Equity of Opportunity is a compilation of 3 different data sets ( Decennial Census, Federal income tax returns, and the American Community Surveys). The data uses \textbf{3224} counties across the U.S. and children's outcomes in adulthood within them. The American Time Use survey comes from randomly selected households who have taken part in the Current Population Survey and breaks down how they spend their day. 

\subsection{The Equity of Opportunity}
 The Equity of Opportunity, a credible source and a collaboration between Harvard, Brown, and the U.S. Census Bureau. The data set is generated by taking information from three sources, them being. The 2000 and 2010 Decennial Census short form, the federal income tax returns from 1989, 1994, 1995, and 1998-2015, as well as the 2000 Decennial Census long form and 2005-2015 American Community Surveys.(depends on the specific variable, this ones comes from federal income tax records from the year 2014-2015). the data covers 3224 counties throughout the United States, but we aggregated that up to the state level.The data covers Cross sections from 1978-2019, but we specifically used the 2014 cross section.
 Our observations the average of respondents in each US county, and the scope is all counties within the U.S. There are 20 variables with some examples being County identification, rank in national income of children, collage attendance by children, and state identification. There's a total of 28,242 Observations from 51 States and territories 

    
\subsection{The American Time Use Survey}
The American Time Use Survey, is take by the U.S. Bureau of Labor Statistics. This data set was generated by randomly chosen households who had completed their eight month of interviews with the Current Population Survey, then given an interview on how they spent their past day within a time-use survey. The coverage of the data is All households within the U.S. of at least 15 years of age, The years are from 2013-14 and it covered individuals willing to particiapte in the survey. There are 792 vairables covered in the stata dataset. The scope is the The United States and there are 40,903 from 51 states and territories


\section{Data Acquisition}
\label{sec:theory}


To Access \textbf{The Equality of Opportunity}, First click the link to travel to the Equality of Opportunity website\href{http://www.equality-of-opportunity.org/data/}{The Equality of Opportunity}, Then scroll down to the section that says "Is the United States Still a Land of Opportunity? Recent Trends in Intergenerational Mobility", then click to view data and download the "Intergenerational Mobility Estimates by County and Birth Cohort" as a Stata download.To access \textbf{The American Time Use Survey} First, use the American Time Use Survey link provided: \href{https://www.bls.gov/tus/}{American Time Use Survey}. From there click on the tab ATUS Data and select the drop down, ATUS data files. From there Select American Time Use Survey - 2014 Microdata files and scroll down until you see ATUS 2014 Activity file (zip) where you can download the code as a .do file for Stata analysis. 
\noindent We have stored the data in GitHub under multiple links, The Equality of opportunity data is stored on GitHub, under the name "countycohorttrends.dta"\href{https://github.com/ecn310/course-project-parent/blob/main/Working%20Data/county_cohort_trends.dta}{GitHub link} as well as \href{https://github.com/ecn310/course-project-parent/issues/5#issuecomment-3463603705}{ATUS Data}Our merged data set is stored on GitHub under \href{https://github.com/ecn310/course-project-parent/blob/c770d1f2cfe64713b70f611617336d8dd344906a/Working%20Data/ATUS%20Fully%20Merged/merged_data.zip}{Merged data}


\section{Data Manipulation}
\label{sec:data}


We have Compiled all the Equity of Opportunity income data from 2014 into Stata. Organized ATUS data and appended all of the stata microfiles together along with the CPS data which has participants identify their state Fips code, we then used the Fips code to merge the ATUS data along with the Eauality of opportunity data because it also contains the Fips code for the state that each county is in. We had to take the mean of each county in each state to be able to merge the Equality of Opportunity data along with the ATUS data. What we ended up with was a fully combined data set with the vairable of child income at age 24 for children born in the 25th percentile as our indicator of intergenerational mobility in each state.

\section{Linking Datasets}
\label{sec:discussion}

We merged our data by Using the statefips variable which shows the federal fips code of each state and after taking the mean of every county in the Equality of Opportunity data set, we used the merge command on Stata to merge the combined ATUS data set with the Equality of Opportunity data set using the statefips variable.


\section{Key Variables}
\label{sec:result}
Some key variables from the ATUS data set are: Weekly income per individual (trernwa) and total time in day spent on secondary childcare with household children (trthh). Some key Equality of Opportunity vairables are: Amount of income (in thousands) that a resident born in the 25th percentile of income has at age 24 (permresp25kr24) and State federal fips code (statefips).

\section{Results}
\label{sec:result}

The analysis we did was between our established variables for parental time investments,trthh, which recorded the amount of time a parent spent in their daily diary entry for childcare and our variable for intergenerational mobility, permresp25kr24, which measures the amount of income (in thousands) that a resident born in the 25th percentile of income has at age 24 aggregated to the state level. We also used a statefips variable to measure which state each observation belonged to. Our research hypothesis is Parents in U.S. states with higher rates of intergenerational mobility invest more time in their children compared to parents in states with lower rates of intergenerational mobility and the null hypothesis we have is there is no relationship between intergenerational mobility of states and parental time investments. In order to merge the two data sets we had, we needed to take the mean of our permresp25kr24 variable by state and we also took the mean of our trthh variable by state to compare the two vairables. We then had to create a new variable of high mobility, hmob, which divided our data into the 26 highest states for time investments to prepare our data for a ttest. We then performed a two sample ttest on our data and found that the difference in means of the higher parental investment states vs the lower parental investment states was 18.48 minutes per day. However, our p value was .0078 and our confidence interval of (-39.08, 2.13) contains zero so we cannot be certian this differnece is not purely due to chance. A graph of the parental investment vs intergenerational mobility that shows each statefips code is also generated through the .do file found on our GitHub repository. This same .do file also contains the code to recreate the ttest performed on the data. Through this ttest, we find that the data we have does not show a statistically significant relationship between parental investment and intergenerational mobility by state, and we fail to reject the null hypothesis. 


\begin{figure}[htbp]
    \centering
    \includegraphics[width=\textwidth]{t_test1.pdf}
    \caption{Intergenerational mobility vs parental investment}
    \textbf{Notes}: This table shows the average minutes spent in childcare of the 26 states above or equal to the median number of minutes compared with the 25 states below the median number of minutes and also has the difference and p value. It is essentially a summarized output of the ttest.
\end{figure}

AI disclosure: Generative AI was not used in the completion of any of the work of the results section. 


\section{discussion}
\label{sec:discussion}

Optional. This is where you would discuss any of the following if they are extensive; otherwise, this material can be put in the Results (caveats) and Conclusions (future work and next steps) sections.
\begin{itemize}
    \item caveats (are there problems with the data that there are no obvious ways to resolve? if so, how might this impact your results?)
    \item future work / next steps
    \item implications of the results: that is, how your findings -- if they were causally identified -- might inform policymaking, etc.
\end{itemize}

\section{Conclusion}
\label{sec:conclusion}

Re-state (in different words) what you did and what you learned. If your discussion (Section 6) would be short, you can just have a Conclusion section that includes your discussion (that is, leave out a separate Discussion section).

\newpage
\section*{Bibliography}
\singlespacing
\setlength\bibsep{0pt}

You can either explicitly include your list of references, or you can learn to use BibTex so that it includes the references automatically.

Either way, this list should include ONLY the papers (reports, book chatpers, etc.) that you actually cite in the text (no extra).

At the same time EVERYTHING you cite in the main text must have an entry here (no references in text that don't have something here).

You can choose which citation style to follow. Whichever you choose, you must follow it consistently, including the convention to always alphabetize by the first author's last name.

\newpage
\section*{Data Appendix} \label{sec:appendixa}
\addcontentsline{toc}{section}{Appendix A}

You should at least direct your reader to your replication package. You might put key elements of your replication package in this section as well.

\end{document}
