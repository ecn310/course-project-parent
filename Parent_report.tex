\documentclass[12pt]{article}

% set margins and spacing
\addtolength{\textwidth}{1.3in}
\addtolength{\oddsidemargin}{-.65in} %left margin
\addtolength{\evensidemargin}{-.65in}
\setlength{\textheight}{9in}
\setlength{\topmargin}{-.5in}
\setlength{\headheight}{0.0in}
\setlength{\footskip}{.375in}
\renewcommand{\baselinestretch}{1.0}
\linespread{1.0}

% load miscellaneous packages
\usepackage{csquotes}
\usepackage[american]{babel}
\usepackage[usenames,dvipsnames]{color}
\usepackage{graphicx,amsbsy,amssymb, amsmath, amsthm, MnSymbol,bbding,times, verbatim,bm,pifont,pdfsync,setspace,natbib}

% enable hyperlinks and table of contents
\usepackage[pdftex,
bookmarks=true,
bookmarksnumbered=false,
pdfview=fitH,
bookmarksopen=true,hyperfootnotes=false]{hyperref}

% define environments
\newtheorem{definition}{Definition}
\newtheorem{fact}{Fact}
\newtheorem{result}{Result}
\newtheorem{proposition}{Proposition}



\begin{document}
\title{ECN 310: Parent}
\author{Aisling Gilmartian\thanks{Akgilmar@syr.edu} \and Devon Mitchell\thanks{Djmitche@syr.edu} \and Yvonne Achancho\thanks{Yeachanc@syr.edu} \and Oliver Levesque\thanks{Oalevesq@syr.edu}}
\date{\vskip-.1in \today}
\maketitle

\vskip.3in
\begin{center} {\bf Abstract} \end{center}

\begin{quote}
{\small Insert abstract text here: 75-200 words, very high-level summary of your project. Look at the three most closely-related papers in your literature review to get an idea of what should be included.}
\end{quote}

\bigskip
\section{Introduction} \label{sec:introduction}

Answer the questions
\begin{enumerate}
    \item \textbf{Why should the reader care? / Why is the topic important?} (required)
    \item \textbf{What question will you answer? How will you do it?} (required)
        \begin{enumerate}
            \item If your theory/hypothesis fit in one paragraph, include it here. If it is longer, make it a separate section after the lit review. EITHER OPTION IS FINE as long as the length is sufficient/appropriate for your project.
        \end{enumerate}
    \item \textbf{What did you find?} (required)
    \item \textbf{Give a "road map" of the paper. Where will the reader find the various parts of your work?} (required)
\end{enumerate}

\section{Literature Review} \label{sec:literature}

Discuss at least seven papers that are closely related to your results (more is better). Explain how they're related. Did you find something similar, or different? Did you look at a different context? Different time period? Different level of detail?
\begin{itemize}
    \item The most important thing to do here is to highlight how what you do compares to these other papers
\end{itemize}

\section{Theoretical Analysis}
\label{sec:theory}
Optional--may include in intro if it's short.


\section{Data}
\label{sec:data}

Describe your data. Where you got it from, how it was generated, what variables you'll use, any major steps you had to take (like merging two data sources together).

In a published paper, a lot of this detail will be in a data appendix. For the purposes of this report, you can include it all here unless it's really complicated, in which case you can put the details in your data appendix (this may be the longest section of your report).

At the very least, your data appendix should explain where your processed data, code and documentation is stored; you must reference that appendix from this section so the reader knows where to look at to get more details.

\subsection{Insert high-level reference to your first data set}

Make subsections if your data comes from more than one source and this section becomes longer than three or so paragraphs. If your data is very straightforward and you only have one data source, you may not need any subsections.

\section{Results}
\label{sec:result}

The analysis we did was between our established variables for parental time investments,trthh, which recorded the amount of time a parent spent in their daily diary entry for childcare and our variable for intergenerational mobility, permresp25kr24, which measures the amount of income (in thousands) that a resident born in the 25th percentile of income has at age 24 aggregated to the state level. We also used a statefips variable to measure which state each observation belonged to. Our research hypothesis is Parents in U.S. states with higher rates of intergenerational mobility invest more time in their children compared to parents in states with lower rates of intergenerational mobility and the null hypothesis we have is there is no relationship between intergenerational mobility of states and parental time investments. In order to merge the two data sets we had, we needed to take the mean of our permresp25kr24 variable by state and we also took the mean of our trthh variable by state to compare the two vairables. We then had to create a new variable of high mobility, hmob, which divided our data into the 26 highest states for time investments to prepare our data for a ttest. We then performed a two sample ttest on our data and found that the difference in means of the higher parental investment states vs the lower parental investment states was 18.48 minutes per day. However, our p value was .0078 and our confidence interval of (-39.08, 2.13) contains zero so we cannot be certian this differnece is not purely due to chance. A graph of the parental investment vs intergenerational mobility that shows each statefips code is also generated through the .do file found on our GitHub repository. This same .do file also contains the code to recreate the ttest performed on the data. Through this ttest, we find that the data we have does not show a statistically significant relationship between parental investment and intergenerational mobility by state, and we fail to reject the null hypothesis. 
\begin{table}[htbp]
\caption{\textbf{Intergenerational Mobility Vs. Average Time spent on childcare}}
\label{tab:mlogit}
\scriptsize
\begin{center}
\input{}
\end{center}
\flushleft
\begin{footnotesize}
\begin{singlespace}
\textbf{Notes}: This table shows the average minutes spent in childcare of the 26 states above or equal to the median number of minutes compared with the 25 states below the median number of minutes and also has the difference and p value. It is essentially a summarized output of the ttest.
\end{singlespace}
\end{footnotesize}
\end{table}
AI disclosure: Generative AI was not used in the completion of any of the work of the results section. 



\section{Discussion}
\label{sec:discussion}

Optional. This is where you would discuss any of the following if they are extensive; otherwise, this material can be put in the Results (caveats) and Conclusions (future work and next steps) sections.
\begin{itemize}
    \item caveats (are there problems with the data that there are no obvious ways to resolve? if so, how might this impact your results?)
    \item future work / next steps
    \item implications of the results: that is, how your findings -- if they were causally identified -- might inform policymaking, etc.
\end{itemize}

\section{Conclusion}
\label{sec:conclusion}

Re-state (in different words) what you did and what you learned. If your discussion (Section 6) would be short, you can just have a Conclusion section that includes your discussion (that is, leave out a separate Discussion section).

\newpage
\section*{Bibliography}
\singlespacing
\setlength\bibsep{0pt}

You can either explicitly include your list of references, or you can learn to use BibTex so that it includes the references automatically.

Either way, this list should include ONLY the papers (reports, book chatpers, etc.) that you actually cite in the text (no extra).

At the same time EVERYTHING you cite in the main text must have an entry here (no references in text that don't have something here).

You can choose which citation style to follow. Whichever you choose, you must follow it consistently, including the convention to always alphabetize by the first author's last name.

\newpage
\section*{Data Appendix} \label{sec:appendixa}
\addcontentsline{toc}{section}{Appendix A}

You should at least direct your reader to your replication package. You might put key elements of your replication package in this section as well.

\end{document}
