\documentclass[12pt]{article}

% set margins and spacing
\addtolength{\textwidth}{1.3in}
\addtolength{\oddsidemargin}{-.65in} %left margin
\addtolength{\evensidemargin}{-.65in}
\setlength{\textheight}{9in}
\setlength{\topmargin}{-.5in}
\setlength{\headheight}{0.0in}
\setlength{\footskip}{.375in}
\renewcommand{\baselinestretch}{1.0}
\linespread{1.0}

% load miscellaneous packages
\usepackage{csquotes}
\usepackage[american]{babel}
\usepackage[usenames,dvipsnames]{color}
\usepackage{graphicx,amsbsy,amssymb, amsmath, amsthm, MnSymbol,bbding,times, verbatim,bm,pifont,pdfsync,setspace,natbib}

% enable hyperlinks and table of contents
\usepackage[pdftex,
bookmarks=true,
bookmarksnumbered=false,
pdfview=fitH,
bookmarksopen=true,hyperfootnotes=false]{hyperref}



\begin{document}
\title{ECN 310: Parent}
% add a fourth name if you have four team members; fill in at least first names below
\author{Aisling Gilmartin\thanks{Akgilmar@syr.edu} \and Devon Mitchell\thanks{Djmitche@syr.edu} \and Yvonne Achancho\thanks{Yeachanc@syr.edu} \and Oliver Levesque\thanks{Oalevesq@syr.edu}}
\date{\vskip-.1in \today}
\maketitle

\vskip.3in

\section{Research Question} \label{sec:question}

Do parents invest more time in their kids in U.S. counties where there is more inequality and/or less inter-generational mobility?

\section{Data Overview} \label{sec:literature}

The Opportunity Atlas is a compilation of 3 different data sets ( Decennial Census, Federal income tax returns, and the American Community Surveys). The data uses \textbf{3224} counties across the U.S. and children's outcomes in adulthood within them. The American Time Use survey comes from randomly selected households ( Specifically Married Parents) who have taken part in the Current Population Survey and breaks down how they spend their day. 

\subsection{The Opportunity Atlas}
\begin{itemize}
  \item What is the source of the data? Opportunity Atlas, a credible source and a collaboration between Harvard, Brown, and the U.S. Census Bureau
  \item How was it generated? The information was taken from three sources, them being. The 2000 and 2010 Decennial Census short form, the federal income tax returns from 1989, 1994, 1995, and 1998-2015, as well as the 2000 Decennial Census long form and 2005-2015 American Community Surveys.( depends on the specific variable, this ones comes from federal income tax records from the year 2014-2015) 
  
  \item What is the coverage of the data?  3224 counties throughout the United States
    \begin{itemize}
        \item What years? Cross sections from 1989-2015
        \item What kind of observations (countries? individual survey respondents? etc) Each row is a county within the United States
        \item Scope All counties within the U.S.  
    \end{itemize}
  \item Number of variables: 31 
    \begin{itemize}
        \item Types of variables: Children outcome in adulthood
    \end{itemize}
  \item How many observations total? (e.g., 20 countries with 10 variables each, for a total of 100 observations)
  3224 counties with 31 variables per
\end{itemize}
    
\subsection{Data Set 2 (The American Time Use Survey)}
    \begin{itemize}
        \item What is the source of the data?
        The American Time Use Survey, is take by the U.S. Bureau of Labor Statistics
    \item How was this dataset generated? Randomly chosen households who had completed their eight month of interviews with the Current Population Survey, then given an interview on how they spent their past day within a time-use survey. 
    \item What is the coverage of the data? All households within the U.S. of at least 15 years of age
            \item What years? 2003-2024
            \item What kind of observations and coverage (countries? individual survey respondents? etc) 
            Married mothers/fathers and their employment status (For our specific use) 
            
            \item Scope? The United States 
            \end{itemize}


\section{Data Acquisition}
\label{sec:theory}

Explain how someone can acquire each dataset (you may need to use subsections again).
\begin{itemize}
    \item \textbf{The Opportunity Atlas} First, click the link \href{https://www.opportunityatlas.org/}{The Opportunity Atlas }. From there select the neighborhood section with mobility outcomes. Select explore the data. On the left side of the screen there will be a drop-down section at the bottom that says download the data. From here select the variable Household Income at Age 35, counties for geography, and all for parent income. The finally hit the download button.
    \item \textbf{The American Time Use Survey} First, search the American Time Use Survey in a search engine to get this link \href{https://www.bls.gov/tus/}{American Time Use Survey}. From there click on the tab ATUS Data and select the drop down, tables. Select the section married parents Table A-6 and Table A-7 and download years 2021-2024 in either HTML or PDF
\end{itemize}

\noindent Where have YOU stored the data? Include link(s).
\begin{itemize}
    \item The data is stored on GitHub, under the name "Household Income at 35"\href{https://github.com/ecn310/course-project-parent/blob/main/cty_kfr_rP_gP_pall.csv#L3210}{GitHub link} as well as \href{https://github.com/ecn310/course-project-parent/issues/5#issuecomment-3378851020}{ATUS Data}
\end{itemize}


\section{Data Manipulation}
\label{sec:data}

What steps have you taken / do you plan to take to get the data ready for analysis? 
\begin{itemize}
    \item Get rid of the counties/variables that have no data tied to them, but still count to the total. Combine the different variables into one major data set rather than each variable having its own individual data set ( due to the format of the website). 
    \item Compile the two tables into one data set, as well cut out non relevant variables 
\end{itemize}

\section{Linking Datasets}
\label{sec:discussion}

How you will link different data sets together (if you'll need to)
\begin{itemize}
    \item What variable(s) in each dataset will you use to merge the datasets together?
       For the OA the variables, High school Graduation Rate, Percent staying in the same tract as adults for children, and Average credit score. As for the ATUS we will use the variables of married parents and their time spent with children on education related activities ( based on parental employment status) 
\end{itemize}


\section{Key Variables}
\label{sec:result}

List of key variables you may use for testing your hypothesis, separately for each dataset

\begin{itemize} 
    \item High school Graduation Rate
    \item Percent staying in the same tract as adults for children,
    \item Average credit score
    \item Time spend on education related activities with child 
\end{itemize}


\end{document}
