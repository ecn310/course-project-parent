\documentclass[12pt]{article}

% set margins and spacing
\addtolength{\textwidth}{1.3in}
\addtolength{\oddsidemargin}{-.65in} %left margin
\addtolength{\evensidemargin}{-.65in}
\setlength{\textheight}{9in}
\setlength{\topmargin}{-.5in}
\setlength{\headheight}{0.0in}
\setlength{\footskip}{.375in}
\renewcommand{\baselinestretch}{1.0}
\linespread{1.0}

% load miscellaneous packages
\usepackage{csquotes}
\usepackage[american]{babel}
\usepackage[usenames,dvipsnames]{color}
\usepackage{graphicx,amsbsy,amssymb, amsmath, amsthm, MnSymbol,bbding,times, verbatim,bm,pifont,pdfsync,setspace,natbib}

% enable hyperlinks and table of contents
\usepackage[pdftex,
bookmarks=true,
bookmarksnumbered=false,
pdfview=fitH,
bookmarksopen=true,hyperfootnotes=false]{hyperref}



\begin{document}
\title{ECN 310: Parent}
% add a fourth name if you have four team members; fill in at least first names below
\author{Aisling Gilmartin\thanks{Akgilmar@syr.edu} \and Devon Mitchell\thanks{Djmitche@syr.edu} \and Yvonne Achancho\thanks{Yeachanc@syr.edu} \and Oliver Levesque\thanks{Oalevesq@syr.edu}}
\date{\vskip-.1in \today}
\maketitle

\vskip.3in

\section{Research Question} \label{sec:question}

Do parents invest more time in their kids in U.S. counties where there is more inequality and/or less inter-generational mobility?

\section{Data Overview} \label{sec:literature}

The Opportunity Atlas is a compilation of 3 different data sets ( Decennial Census, Federal income tax returns, and the American Community Surveys). The data uses \textbf{3224} counties across the U.S. and children's outcomes in adulthood within them. The American Time Use survey comes from randomly selected households ( We will be focusing on Specifically Married Parents) who have taken part in the Current Population Survey and breaks down how they spend their day. 

\subsection{The Opportunity Atlas}
\begin{itemize}
  \item What is the source of the data? Opportunity Atlas, a credible source and a collaboration between Harvard, Brown, and the U.S. Census Bureau
  \item How was it generated? The information was taken from three sources, them being. The 2000 and 2010 Decennial Census short form, the federal income tax returns from 1989, 1994, 1995, and 1998-2015, as well as the 2000 Decennial Census long form and 2005-2015 American Community Surveys.( depends on the specific variable, this ones comes from federal income tax records from the year 2014-2015) 
  
  \item What is the coverage of the data?  3224 counties throughout the United States
    \begin{itemize}
        \item What years? Cross sections from 1978-2019
        \item What kind of observations (countries? individual survey respondents? etc) Our observations are individual survey respondents of all income levels in each of the observed counties
        \item Scope All counties within the U.S.  
    \end{itemize}
  \item Number of variables: 6
    \begin{itemize}
        \item Types of variables: Children outcome in adulthood
    \end{itemize}
  \item How many observations total? (e.g., 20 countries with 10 variables each, for a total of 100 observations)
  3224 counties with 6 variables per
\end{itemize}
    
\subsection{The American Time Use Survey}
    \begin{itemize}
        \item What is the source of the data?
        The American Time Use Survey, is take by the U.S. Bureau of Labor Statistics
    \item How was this dataset generated? Randomly chosen households who had completed their eight month of interviews with the Current Population Survey, then given an interview on how they spent their past day within a time-use survey. 
    \item What is the coverage of the data? All households within the U.S. of at least 15 years of age
            \item What years? 2003-2024
            \item What kind of observations and coverage (countries? individual survey respondents? etc) 
            Married mothers/fathers and their employment status (For our specific use) 
            
            \item Scope? The United States 
            \end{itemize}


\section{Data Acquisition}
\label{sec:theory}

Explain how someone can acquire each dataset (you may need to use subsections again).
\begin{itemize}
    \item \textbf{The Opportunity Atlas} First click the link to travel to the opportuinity atalas site with data \href{http://www.equality-of-opportunity.org/data/}{The Equality of Opportunity}, Then scroll down to the section that says "Is the United States Still a Land of Opportunity? Recent Trends in Intergenerational Mobility", then click on it and download the "Intergenerational Mobility Estimates by County and Birth Cohort" as a Stata download.
    \item \textbf{The American Time Use Survey} First, search the American Time Use Survey in a search engine to get this link \href{https://www.bls.gov/tus/}{American Time Use Survey}. From there click on the tab ATUS Data and select the drop down, ATUS data files. From there Select 2014 as your micro file year and scroll down until you see ATUS 2014 Activity file (zip) where you can download the data as a .do file for Stata analysis. 
\end{itemize}

\noindent Where have YOU stored the data? Include link(s).
\begin{itemize}
    \item The data is stored on GitHub, under the name "countycohorttrends.dta"\href{https://github.com/ecn310/course-project-parent/blob/main/Working%20Data/county_cohort_trends.dta}{GitHub link} as well as \href{https://github.com/ecn310/course-project-parent/issues/5#issuecomment-3463603705}{ATUS Data}
\end{itemize}


\section{Data Manipulation}
\label{sec:data}

What steps have you taken / do you plan to take to get the data ready for analysis? 
\begin{itemize}
    \item Compiled all the Opportunity Atlas income data from 2014 into excel and onto stata. Organised ATUS data and will begin data analysis soon. 
    \item Need to clean and break down excel data on Stata and perform statistical analysis.
    \item need to merge the two data sets into one on Stata and perform analysis for next week
\end{itemize}

\section{Linking Datasets}
\label{sec:discussion}

How you will link different data sets together (if you'll need to)
\begin{itemize}
    \item What variable(s) in each dataset will you use to merge the datasets together?
       We need to compile the two data sets together into Stata and use the merge command but we are currently having trouble getting all of our data into a like file type to be able to perform the merger.
\end{itemize}


\section{Key Variables}
\label{sec:result}

List of key variables you may use for testing your hypothesis, separately for each dataset

\begin{itemize} 
    \item Income per household
    \item Child income by parental financial disposition; born in 1978 (27yrs)
    \item Time spend on child raring related activities with child 
    \item Population of each county
    \item Average income by county
\end{itemize}


\end{document}