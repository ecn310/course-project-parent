\documentclass[12pt]{article}

% set margins and spacing
\addtolength{\textwidth}{1.3in}
\addtolength{\oddsidemargin}{-.65in} %left margin
\addtolength{\evensidemargin}{-.65in}
\setlength{\textheight}{9in}
\setlength{\topmargin}{-.5in}
\setlength{\headheight}{0.0in}
\setlength{\footskip}{.375in}
\renewcommand{\baselinestretch}{1.0}
\linespread{1.0}

% load miscellaneous packages
\usepackage{csquotes}
\usepackage[american]{babel}
\usepackage[usenames,dvipsnames]{color}
\usepackage{graphicx,amsbsy,amssymb, amsmath, amsthm, MnSymbol,bbding,times, verbatim,bm,pifont,pdfsync,setspace,natbib}

% enable hyperlinks and table of contents
\usepackage[pdftex,
bookmarks=true,
bookmarksnumbered=false,
pdfview=fitH,
bookmarksopen=true,hyperfootnotes=false]{hyperref}



\begin{document}
\title{ECN 310: Parent}
% add a fourth name if you have four team members; fill in at least first names below
\author{Aisling Gilmartin\thanks{Akgilmar@syr.edu} \and Devon Mitchell\thanks{Djmitche@syr.edu} \and Yvonne Achancho\thanks{Yeachanc@syr.edu} \and Oliver Levesque\thanks{Oalevesq@syr.edu}}
\date{\vskip-.1in \today}
\maketitle

\vskip.3in

\section{Research Question} \label{sec:question}

Do parents invest more time in their kids in U.S. states where there is more inequality and/or less inter-generational mobility?

\section{Data Overview} \label{sec:literature}

The Equity of Opportunity is a compilation of 3 different data sets ( Decennial Census, Federal income tax returns, and the American Community Surveys). The data uses \textbf{3224} counties across the U.S. and children's outcomes in adulthood within them. The American Time Use survey comes from randomly selected households who have taken part in the Current Population Survey and breaks down how they spend their day. 

\subsection{The Equity of Opportunity}
\begin{itemize}
  \item What is the source of the data? The Equity of Opportunity, a credible source and a collaboration between Harvard, Brown, and the U.S. Census Bureau
  \item How was it generated? The information was taken from three sources, them being. The 2000 and 2010 Decennial Census short form, the federal income tax returns from 1989, 1994, 1995, and 1998-2015, as well as the 2000 Decennial Census long form and 2005-2015 American Community Surveys.(depends on the specific variable, this ones comes from federal income tax records from the year 2014-2015) 
  
  \item What is the coverage of the data?  3224 counties throughout the United States, but we aggregated that up to the state level
    \begin{itemize}
        \item What years? Cross sections from 1978-2019
        \item What kind of observations (countries? individual survey respondents? etc) Our observations are individual survey respondents of all income levels in each of the observed counties
        \item Scope All counties within the U.S.  
    \end{itemize}
  \item Number of variables: 20
    \begin{itemize}
        \item Types of variables: County identification, rank in national income of children, collage attendance by children, state identification
    \end{itemize}
  \item How many observations total? (e.g., 20 countries with 10 variables each, for a total of 100 observations)
  28,242 Observations from 51 States and territories 
\end{itemize}
    
\subsection{The American Time Use Survey}
    \begin{itemize}
        \item What is the source of the data?
        The American Time Use Survey, is take by the U.S. Bureau of Labor Statistics
    \item How was this dataset generated? Randomly chosen households who had completed their eight month of interviews with the Current Population Survey, then given an interview on how they spent their past day within a time-use survey. 
    \item What is the coverage of the data? All households within the U.S. of at least 15 years of age
            \item What years? 2003-2024
            \item What kind of observations and coverage (countries? individual survey respondents? etc) 
            Married mothers/fathers and their employment status (For our specific use) 
            \item Number of variables: 792
            \item Scope? The United States 
            \item Number of observations: 40,903 from 51 states and territories
            \end{itemize}


\section{Data Acquisition}
\label{sec:theory}

Explain how someone can acquire each dataset (you may need to use subsections again).
\begin{itemize}
    \item \textbf{The Equality of Opportunity} First click the link to travel to the Equality of Opportunity website\href{http://www.equality-of-opportunity.org/data/}{The Equality of Opportunity}, Then scroll down to the section that says "Is the United States Still a Land of Opportunity? Recent Trends in Intergenerational Mobility", then click to view data and download the "Intergenerational Mobility Estimates by County and Birth Cohort" as a Stata download.
    \item \textbf{The American Time Use Survey} First, use the American Time Use Survey link provided: \href{https://www.bls.gov/tus/}{American Time Use Survey}. From there click on the tab ATUS Data and select the drop down, ATUS data files. From there Select American Time Use Survey - 2014 Microdata files and scroll down until you see ATUS 2014 Activity file (zip) where you can download the code as a .do file for Stata analysis. 
\end{itemize}

\noindent Where have YOU stored the data? Include link(s).
\begin{itemize}
    \item The data is stored on GitHub, under the name "countycohorttrends.dta"\href{https://github.com/ecn310/course-project-parent/blob/main/Working%20Data/county_cohort_trends.dta}{GitHub link} as well as \href{https://github.com/ecn310/course-project-parent/issues/5#issuecomment-3463603705}{ATUS Data}
    Our merged data set is stored on GitHub under \href{https://github.com/ecn310/course-project-parent/blob/c770d1f2cfe64713b70f611617336d8dd344906a/Working%20Data/ATUS%20Fully%20Merged/merged_data.zip}{Merged data}
\end{itemize}


\section{Data Manipulation}
\label{sec:data}


\begin{itemize}
    \item Compiled all the Equity of Opportunity income data from 2014 into Stata. Organized ATUS data and appended all of the stata microfiles together along with the CPS data which has participants identify their state Fips code, we then used the Fips code to merge the ATUS data along with the Eauality of opportunity data because it also contains the Fips code for the state that each county is in. We had to take the mean of each county in each state to be able to merge the Equality of Opportunity data along with the ATUS data. What we ended up with was a fully combined data set with the vairable of child income at age 24 for children born in the 25th percentile as our indicator of intergenerational mobility in each state.
\end{itemize}

\section{Linking Datasets}
\label{sec:discussion}

\begin{itemize}
    \item 
    We Used the statefips variable which shows the federal fips code of each state and after taking the mean of every county in the Equality of Opportunity data set, we used the merge command on stata to merge the combined ATUS data set with the Equality of Opportunity data set using the statefips variable.
\end{itemize}


\section{Key Variables}
\label{sec:result}

ATUS variables
\begin{itemize} 
    \item Weekly income per individual (trernwa)
    \item total time in day spent on secondary childcare with household children (trthh)
\end{itemize}
Opportunity Atlas variables
\begin{itemize} 
    \item Amount of income (in thousands) that a resident born in the 25th percentile of income has at age 24 (permresp25kr24)
    \item State federal fips code (statefips)
\end{itemize}


\end{document}